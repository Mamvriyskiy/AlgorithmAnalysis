\chapter*{Введение} 
\addcontentsline{toc}{chapter}{Введение}

\begin{bfseries}Расстояние Левенштейна\end{bfseries} -- минимальное количество операций удаления, вставки одного символа и замены 
одного символа другим, необходимым для превращения одной строки в другую.

Оно применяется в теории информации и компьютерной лингвистике для решения следующих задач:

\begin{itemize}[label=---] % создание списка с маркерами, в данном случае маркер - "---"
    \item исправление ошибок в слове(в поисковых системах, базах данных, при вводе текста, при автоматическом распознавании отсканированного текста или речи);
	\item сравнение текстовых файлов утилитой diff;
	\item для сравнения геномов, хромосом и белков в биоинформатике.
\end{itemize}

Целью данной лабораторной работы является описание и исследование 
алгоритмов поиска расстояний Левенштейна и Дамерау-Левенштейна.

Для достижения поставленнной цели необходимо выполнить следующие задачи:
\begin{enumerate}[label={\arabic*)}] 
	\item Проанализировать алгоритмы поиска расстояний Дамерау-Левенш-
	тейна и Левенштейна;
	\item Создать программное обеспечение, реализующее следующие алгоритмы:
	\begin{itemize}[label=---]
		\item нерекурсивный алгоритм поиска редакционного расстояния Левенштейна;
		\item нерекурсивный алгоритм поиска редакционного расстояния \\
		Дамерау-Левенштейна;
		\item рекурсивный алгоритм поиска редакционного расстояния \\
		Дамерау-Левенштейна;
		\item рекурсивный с кэшированием алгоритм поиска редакционного расстояния
		Дамерау-Левенштейна.
	\end{itemize}
	\item Выбрать инструменты для замера процессорного времени выполнения реализаций алгоритмов;
	\item Провести анализ затрат реализаций алгоритмов по времени и по памяти, определить влияющие на них характеристики.
\end{enumerate}
