\chapter*{Введение} % Не добавляется номер загаловка

\begin{bfseries}Расстояние Левенштейна\end{bfseries} - минимальное количество операций вставки, удаления одного символа и замены 
одного символа другим, необходимым для превращения одной строки в другую.

Оно применяется в теории информации и компьютерной лингвистике для решения следующих задач:

\begin{itemize}[label=---] % создание списка с маркерами, в данном случае маркер - "---"
    \item исправление ошибок в слове(в поисковых системах, базах данных, при вводе текста, при автоматическом распознавании отсканированного текста или речи);
	\item сравнение текстовых файлов утилитой diff;
	\item для сравнения геномов, хромосом и белков в биоинформатике.
\end{itemize}


Цели данной лабораторной работы:
\begin{enumerate}[label={\arabic*)}] % Нумерованный список
	\item Описать алгоритмы поиска расстояний Левенштейна и Дамерау-Левен- штейна.
	\item Создать программное обеспечение, реализующее следующие алгоритмы:
	\begin{itemize}[label=---]
		\item нерекурсивный алгоритм поиска расстояния Левенштейна;
		\item нерекурсивный алгоритм поиска расстояния Дамерау-Левенштейна;
		\item рекурсивный алгоритм поиска расстояния Дамерау-Левенштейна;
		\item рекурсивный с кешированием алгоритм поиска расстояния Дамерау-Левенштейна.
	\end{itemize}
	\item Выбрать инструменты для замера процессорного времени выполнения реализаций алгоритмов.
	\item Провести анализ затрат реализаций алгоритмов по времени и по памяти, определить влияющие на них характеристики.
\end{enumerate}

