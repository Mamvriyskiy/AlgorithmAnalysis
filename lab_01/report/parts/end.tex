\chapter{Заключение}

В ходе проделанной работы был изучен метод динамического программирования 
на материале реализации алгоритмов нахождения расстояния Левенштейна и 
Дамерау-Левенштейна. Также были изучены алгоритмы поиска расстояния Левенштейна и 
Дамерау-Левенштейна нахождения расстояния между строками и получены практические
навыки раелизации указанных алгоритмов в матричной и рекурсивных версиях.

Экспериментально было подтверждено различие во временной эффективности рекурсивной
и нерекурсивной реализаций выбранного алгоритма определения расстояния между 
строками при помощи разработаного программного обеспечения на материале замеров
процессорного времени выполнения реализации на варьирующихся длинах строк.
Рекурсивная реализация алгоритма Левенштейна проигрывает нерекурсивной по времени
исполнения в несколько десятков раз. Так же стоит отметить, что итеративный алгоритм
Левештейна выполняется немного быстрее, чем итеративный алгоритм Дамерау - Левенштейна,
но в целом алгоритмы выполняются за примерно одинаковое время.

Теоретически было рассчитано использования памяти в каждой из реализаций алгоритмов 
нахождения расстояния Левенштейна и Дамерау - Левенштейна. Рекурсивный алгоритм с 
мемоизацией в несколько десятков раз больше памяти, чем итеративная реализация 
алгоритма нахождения расстояния Левенштейна, из-за рекурсивного копирования 
вспомогательной матрицы.
