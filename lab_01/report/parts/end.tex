\chapter*{Заключение}
\addcontentsline{toc}{chapter}{Заключение}

В результате исследования было определено, что лучшие показатели по времени дает 
матричная реализация алгоритма нахождения расстояния 
Левенштейна, которая немного выигрывает у алгоритма нахождения
расстояния Дамерау-Левенштейна за счет отсутствия дополнительной проверки.
Худшие показатели у рекурсивного  алгоритма нахождения расстояния 
Левенштейна, который проигрывает другим алгоритмам примерно в 
14 -- 40 тыс. раз. При этом итеративные реализации с использованием матрицы и рекурсивная с использованием кэша
занимают примерно в 10 -- 56 раз больше памяти, чем рекурсивная.



Цель, которая заключается в описании и исследовании алгоритмов поиска расстояний Левенштейна и Дамерау-Левенштейна, выполнена, также в ходе выполнения лабораторной 
работы были решены следующие задачи:
\begin{enumerate}[label={\arabic*)}] 
	\item были описаны алгоритмы поиска расстояний Левенштейна и Дамерау-Левенштейна;
	\item создано программное обеспечение, реализующее следующие алгоритмы:
	\begin{itemize}[label=---]
		\item нерекурсивный алгоритм поиска расстояния Левенштейна;
		\item нерекурсивный алгоритм поиска расстояния Дамерау-\\
		Левенштейна;
		\item рекурсивный алгоритм поиска расстояния Дамерау-Левенштейна;
		\item рекурсивный с кешированием алгоритм поиска расстояния \\
		Дамерау-Левенштейна.
	\end{itemize}
	\item выбраны инструменты для замера процессорного времени выполнения реализаций алгоритмов;
	\item проведен анализ затрат реализаций алгоритмов по времени и по памяти, определены влияющие на них характеристики.
\end{enumerate}
