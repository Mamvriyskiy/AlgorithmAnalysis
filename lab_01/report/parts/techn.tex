\chapter{Технологический раздел}

\section{Требования к ПО}
\textbf{Требования к вводу:}
\begin{enumerate}
	\item На вход подаются две строки в любой раскладке (в том числе и пустые);
	\item ПО должно выводить полученное расстояние;
\end{enumerate}

\section{Средства реализации}
Для реализации программы нахождения расстояние Левенштейна я выбрал язык программирования Golang. Данный
выбор обусловлен моим желанием расширить свои знания в области применения данного язкыа программирования.

\clearpage

\section{Реализация алгоритмов}
В следующих листингах приведена реализация алгоритмов нахождения расстония Левенштейна и Дамерау-Левенштейна.

\begin{lstlisting}[label=lst:LM,caption=Функция нахождения расстояния Левенштейна нерекурсивным способом]
func LM(matrix [][]int, a, b int, s1, s2 []rune) int { 
	for i := 1; i < a; i++ {
		r2 := s2[i - 1]
		for j := 1; j < b; j++ {
			r1 := s1[j - 1]
			if r1 != r2 {
				matrix[i][j] = min(matrix[i][j - 1] + 1, matrix[i - 1][j] + 1, matrix[i - 1][j - 1] + 1)
			} else {
				matrix[i][j] = min(matrix[i][j - 1] + 1, matrix[i - 1][j] + 1, matrix[i - 1][j - 1])
			}
		}
	}

	return matrix[a - 1][b - 1]
}
\end{lstlisting}

\begin{lstlisting}[label=lst:DL,caption=Функция нахождения расстояния Левенштейна-Дамерау с помощью рекурсии]
func DL(s1, s2 []rune) int {
    lenB := len(s1)
    lenA := len(s2)

    dist := 0 

    if lenA == 0 {
        dist = lenB
    } else if lenB == 0 {
        dist = lenA
    } else {
        k := 0
        if s1[lenB - 1] != s2[lenA - 1] {
            k = 1
        }	

        dist = min(DL(s1, s2[ : lenA - 1]) + 1, DL(s1[ : lenB - 1], s2) + 1, DL(s1[ : lenB - 1], s2[ : lenA - 1]) + k)

        if lenB > 1 && lenA > 1 && s1[lenB - 1] == s2[lenA - 2] && s1[lenB - 2] == s2[lenA - 1] {
            dist = min(dist, DL(s1[ : lenB - 2], s2[ : lenA - 2]) + 1)
        }
    }

    return dist
}
\end{lstlisting}

\begin{lstlisting}[label=lst:DLCash,caption=Функция нахождения расстояния Левенштейна-Дамерау с помощью рекурсии и кеша]
func DLCash(matrix [][]int, s1, s2 []rune) int {
    lenB := len(s1)
    lenA := len(s2)

    dist := 0

    if matrix[lenA][lenB] == -1 {
        k := 0
        if s1[lenB - 1] != s2[lenA - 1] {
            k = 1
        }

        dist = min(DLCash(matrix, s1, s2[ : lenA - 1]) + 1, DLCash(matrix, s1[ : lenB - 1], s2) + 1, DLCash(matrix, s1[ : lenB - 1], s2[ : lenA - 1]) + k)

        if lenB > 1 && lenA > 1 && s1[lenB - 1] == s2[lenA - 2] && s1[lenB - 2] == s2[lenA - 1] {
            dist = min(dist, DLCash(matrix, s1[ : lenB - 2], s2[ : lenA - 2]) + 1)
        }

        matrix[lenA][lenB] = dist
    } else {
        dist = matrix[lenA][lenB]
    }

    return dist
}
\end{lstlisting}

\begin{lstlisting}[label=lst:DLM,caption=Функция нахождения расстояния Дамерау-Левенштейна нерекурсивным способом]
func DLM(matrix [][]int, a, b int, s1, s2 []rune) int {
    for i := 1; i < a; i++ {
        for j := 1; j < b; j++ {
            if s1[j - 1] != s2[i - 1] {
                matrix[i][j] = min(matrix[i][j - 1] + 1, matrix[i - 1][j] + 1, matrix[i - 1][j - 1] + 1)
            } else {
                matrix[i][j] = min(matrix[i][j - 1] + 1, matrix[i - 1][j] + 1, matrix[i - 1][j - 1])
            }

            if i > 2 && j > 2 && s1[j - 1] == s2[i - 2] && s1[j - 2] == s2[i - 1] {
                matrix[i][j] = min(matrix[i][j], matrix[i - 2][j - 2] + 1)
            }
        }
    }

    return matrix[a - 1][b - 1]
}
\end{lstlisting}

\clearpage

\section{Тестирование}
\begin{table}[ht]
	\small
	\begin{center}
		\begin{threeparttable}
		\label{tbl:func_tests}
		\begin{tabular}{|c|c|c|c|c|c|}
			\hline
			\multicolumn{2}{|c|}{\bfseries Входные данные}
			& \multicolumn{4}{c|}{\bfseries Расстояние и алгоритм} \\ 
			\hline 
			&
			& \multicolumn{1}{c|}{\bfseries Левенштейна} 
			& \multicolumn{3}{c|}{\bfseries Дамерау-Левенштейна} \\ \cline{3-6}
			
			\bfseries Строка 1 & \bfseries Строка 2 & \bfseries Итеративный & \bfseries Итеративный
			
			& \multicolumn{2}{c|}{\bfseries Рекурсивный} \\ \cline{5-6}
			& & & & \bfseries Без кеша & \bfseries С кешом \\
			\hline
			a & d & 1 & 1 & 1 & 1 \\
			\hline
			a & a & 0 & 0 & 0 & 0 \\
			\hline
			переговоры & перегрел & 5 & 5 & 5 & 5 \\
			\hline
			cat & cats & 1 & 1 & 1 & 1 \\
			\hline
			кот & кто & 2 & 1 & 1 & 1 \\
			\hline
			12345 & 54321 & 4 & 4 & 4 & 4 \\
			\hline
		\end{tabular}	
        \caption{Тестовые данные}
		\end{threeparttable}
	\end{center}
\end{table}

\section*{Вывод}
Были реализованы алгоритмы поиска расстояния Левенштейна итеративно,
а также поиска расстояния Дамерау–Левенштейна итеративно, рекурсивно и рекурсивного с кеширования. Проведено тестирование реализаций алгоритмов.


