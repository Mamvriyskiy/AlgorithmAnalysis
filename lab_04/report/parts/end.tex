\chapter*{Заключение}
\addcontentsline{toc}{chapter}{Заключение}

В ходе исследования был проведен сравнительный анализ алгоритмов, в результате
которого было выяснено, что при движении от минимального количества ядер к оптимальному(8 потоков) время работы алгоритма
уменьшается, при превышении оптимального количества ядер время работы алгоритма начинает замедляться, так как происходят затраты времени
на переключение ядра между потоками.

Исходя из данных показателей для более быстрой работы необходимо использовать оптимальное количество ядер.

Цель данной лабораторной работы, которая заключается в исследовании
параллельного программирования на основе алгоритма
нахождения обратной матрицы выполнены.
Также были выполнены следующие задачи:
\begin{itemize}[left=\parindent]
    \item описаны основы распараллеливания вычислений;
    \item разработано программное обеспечение, реализующее однопоточный алгоритм нахождения обратной матрицы;
    \item разработано программное обеспечение, реализующее многопоточную
        версию данного алгоритма;
    \item выполнены замеры процессорного времени работы алгоритмов;
    \item проведен сравнительный анализ по времени работы данных алгоритмов.
\end{itemize}





