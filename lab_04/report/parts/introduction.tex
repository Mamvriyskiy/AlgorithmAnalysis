\chapter*{Введение}
\addcontentsline{toc}{chapter}{Введение}

Сегодня компьютерам требуется производить все более трудоемкие вычисления на больших
объемах данных. При этом предъявляются требования к скорости вычислений.

Одним из возможных решений уменьшения
скорости выполнения задач является параллельное программирование. На одном
устройстве параллельные вычисления можно организовать с помощью
многопоточности -- способности центрального процессора или одного ядра
в многоядерном процессоре одновременно выполнять несколько процессов или
потоков, соответствующим образом поддерживаемых операционной системой. При
последовательной реализации какого-либо алгоритма, его программу выполняет
только одно ядро процессора. Если же реализовать алгоритм так, что независимые
вычислительные задачи смогут выполнять несколько ядер параллельно, то это
приведет к ускорению решения всей задачи в целом \cite{intro}.

Для реализации параллельных вычислений требуется выделить те участки алгоритма,
которые могут выполняться параллельно без изменения итогового результат, 
также необходимо правильно организовать работу с данными, чтобы не потерять
вычисленные значения.

\textbf{Целью данной работы} является исследование
параллельного программирования на основе алгоритма
нахождения обратной матрицы.

Для поставленной цели необходимо выполнить следующие задачи:
\begin{enumerate}
	\item описать основы распараллеливания вычислений;
	\item разработать программное обеспечение, которое реализует однопоточный алгоритм нахождения обратной матрицы;
	\item разработать и реализовать многопоточную версию данного алгоритма;
	\item выполнить замеры процессорного времени работы алгоритма;
	\item провести сравнительный анализ по времени работы реализаций алгоритма.
\end{enumerate}
