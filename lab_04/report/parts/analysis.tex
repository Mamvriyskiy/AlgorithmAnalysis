\chapter{Аналитическая часть}
В данном разделе представлено теоретическое описание алгоритмов нахождения обратной матрицы.


\section{Алгоритм нахождения обратной матрицы методом Гаусса-Жордана}

Пусть A и B -- квадратные матрицы порядка n и $A*B = E$, тогда матрица B называется 
\textbf{обратной} к A и обозначается как $A^{-1}$, то есть A*$A^{-1}$ = $A^{-1}$*A = E ~\cite{matrix}.

В данной лабораторной работе рассмотрим метод Гаусса-Жордана, целью которого является перевод 
матрицы (A|E) 

\begin{equation}
    \begin{pmatrix}
            -5 & 23 & -24 & \mid & 1 & 0 & 0\\
            -1 & 4 & -5 & \mid & 0 & 1 & 0\\
            9 & -40 & 43 & \mid & 0 & 0 & 1\\
    \end{pmatrix},
\end{equation}

к виду (E|$A^{-1}$)
\begin{equation}
    \begin{pmatrix}
            1 & 0 & 0 & \mid & 14 & \frac{29}{2} & \frac{19}{2} \\
            0 & 1 & 0 & \mid & 1 & \frac{-1}{2}  & \frac{1}{2} \\
            0 & 0 & 1 & \mid & -2 & \frac{-7}{2}  & \frac{-3}{2} \\
    \end{pmatrix}.
\end{equation}

Данный метод заключается в последовательном обходе строк матрицы, то есть:
\begin{enumerate}
    \item на k-м шаге работаем с k-й строкой $r_{k}$, k-ый элемент которой обозначим как $a_{k}$;
    \item если $a_{k}$ = 0, то меняем местами строку $r_{k}$ с одной из тех нижележащих строк, у которых 
    k-ый элемент отличен от нуля. Если таких строк нет, то обратная матрица не существует;
    \item если $a_{k} \neq 1$, умножаем строку $r_{k}$ на $\frac{1}{a_{k}}$, если $a_{k} = 1$, 
    то никакого домножения делать не надо;
    \item с помощью строки $r_{k}$ производим обнуление всех остальных ненулевых элементов k-ого столбца, 
    переходим к следующему шагу.
\end{enumerate} 

После обработки последней строки, матрица до черты станет единичной, алгоритм завершится.~\cite{gauss}

\section{Параллельный алгоритм}

В методе нахождения обратной матрицы Гаусса-Жордана умножение $r_{k}$ строки на $\frac{1}{a_{k}}$ 
и обнуление всех ненулевых жлементов k-ого столбца происходит независимо, поэтому есть
возможность произвести распараллеливание данных вычислений. Количество
строк, на которых производит вычисление один поток, будет определяться
количеством потоков, исходная и обратная матрицы будут храниться в
разделяемой переменной, доступ к которым будут иметь все потоки.

\section*{Вывод}
В данном разделе был рассомотрен алгоритм нахождения обратной матрицы методом Гаусса-Жордана
и описана возможность его распараллеливания.

