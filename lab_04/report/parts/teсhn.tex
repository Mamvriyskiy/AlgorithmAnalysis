\chapter{Технологическая часть}

В данном разделе описаны требования к программному обеспечению, средства
реализации, приведены листинги кода и данные, на которых будет проводиться
тестирование.

\section{Средства реализации}

В качестве языка программирования для реализации данной лабораторной работы 
был выбран язык программирования $C$~\cite{c}. Данный выбор обусловлен наличием у языка 
встроенного модуля для измерения процессорного времени и соответствием с 
выдвинутыми требованиям.

Работа с потоками осуществлялась с помощью функций из модуля <pthread.h>~\cite{pth}. 
Для работы с сущностью вспомогательного потока необходимо воспользоваться 
функцией $pthread\_create()$ для создания потока и указания функции, которую 
начнет выполнять созданный поток. Далее при помощи вызова $pthread\_join()$ 
необходимо (в рамках данной лабораторной работы) дождаться завершения всех
вспомогательных потоков, чтобы в главном потоке обработать результаты их работы.

\clearpage
\section{Реализация алгоритмов}
В данном подразделе представлены листинги кода ранее описанных алгоритмов:
\begin{itemize}[left=\parindent]
    \item алгоритм метода Гаусса-Жордана (листинг \ref{lst:gauss});
    \item параллельный алгоритм метода Гаусса-Жордана (листинг \ref{lst:parallgauss});
    \item алгориты вспомогательных функция (листинг \ref{lst:vspom}).
\end{itemize}

\noindent
\begin{minipage}{\linewidth}
\begin{lstinputlisting}[
	caption={Реализация алгоритма метода Гаусса-Жордана.},
	label={lst:gauss},
	linerange={25-49}
]{../main/gauss.c}
\end{lstinputlisting}
\end{minipage}

\noindent
\begin{lstinputlisting}[
	caption={Реализация параллельного алгоритма метода Гаусса-Жордана.},
	label={lst:parallgauss},
	linerange={8-66}
]{../main/gaussThread.c}
\end{lstinputlisting}

\noindent
\begin{minipage}{\linewidth}
\begin{lstinputlisting}[
	caption={Реализация алгоритмов вспомогательных функций.},
	label={lst:vspom},
	linerange={7-23}
]{../main/gauss.c}
\end{lstinputlisting}
\end{minipage}

\clearpage
\section{Функциональные тесты}

В таблице \ref{tab:tests} приведены функциональные тесты для алгоритмов
нахождения обратной матрицы.

\begin{table}[h!]
	\begin{center}
        \caption{\label{tab:tests}Функциональные тесты}
        \begin{tabular}{|c|c|}
			\hline
            \textbf{Матрица} & \textbf{Ожидаемый
            результат}\\ [2mm]
            \hline
			$\begin{pmatrix}
                5 & 9 & 2\\
                6 & 11 & 4\\
                1 & 3 & 1
			\end{pmatrix}$ &
			$\begin{pmatrix}
                0.1 & 0.3 & -2\\
                0.2 & 0 & 0.9\\
                -1 & 0.7 & 0
			\end{pmatrix}$\\
            \hline
            $\begin{pmatrix}
                1 & 0 & 0 \\
                0 & 2 & 0 \\
                0 & 0 & 3 \\
			\end{pmatrix}$ &
			$\begin{pmatrix}
                1 & 0 & 0 \\
                0 & 0.5 & 0 \\
                0 & 0& 0.33333 \\
			\end{pmatrix}$\\
            \hline
			$\begin{pmatrix}
                2 & 1 & -1 \\
                3 & 2 & -2 \\
                1 & -1 & 2
			\end{pmatrix}$ &
			$\begin{pmatrix}
                2 & -1 & 0 \\
                -8 & 5 & 1 \\
                -5 & 3 & 1
			\end{pmatrix}$\\
            \hline
			$\begin{pmatrix}
                1 & 2\\
                3 & 4\\
			\end{pmatrix}$ &
            $\begin{pmatrix}
                -2 & 1\\
                1.5 & -0.5
			\end{pmatrix}$\\
            \hline
		\end{tabular}
	\end{center}
\end{table}

Алгоритмы, реализованные в данной лабораторной работе, функци-
ональные тесты прошли успешно.

\section*{Вывод}

В данном разделе были реализованы алгоритм нахождения обратой 
матрицы методом Гаусса-Жорадана и его параллельная версия.
Также было выполнено тестирования данных алгоритмов.