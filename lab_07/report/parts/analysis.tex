\chapter{Аналитическая часть}

В данной части будут рассмотрены алгоритмы поиска подстроки в строке.

\section{Стандартный алгоритм}
Стандартный алгоритм начинает работу со сравнения первого символа текста с первым символом подстроки. 
Если они совпадают, то происходит переход ко второму символу текста и подстроки. 
При совпадении сравниваются следующие символы. Так продолжается до тех пор, пока не 
окажется, что подстрока целиком совпала с отрезком текста, или пока не встретятся несовпадающие символы. 
В первом случае задача решена, во втором мы сдвигаем указатель текущего положения в тексте на один символ 
и заново начинаем сравнение с подстрокой~\cite{mak}.

\section{Алгоритм Бойера-Мура}
Алгоритм Бойера-Мура осуществляет сравнение с образцом справа налево, а не слева направо. 
Исследуя искомый образец, можно осуществлять более эффективные прыжки в тексте при обнаружении несовпадения. 
В этом алгоритме кроме таблицы суффиксов применяется таблица стоп-символов. Она заполняется для каждого символа в
алфавите. Для каждого встречающегося в подстроке символа таблица заполняется по принципу максимальной позиции символа 
в строке, за исключением последнего символа. При определении сдвига при очередном несовпадении строк, выбирается 
максимальное значение из таблицы суффиксов и стоп-символов~\cite{mak}.

\section*{Вывод}
В данном разделе были рассмотрены основные алгоритмы поиска подстроки в строке.
