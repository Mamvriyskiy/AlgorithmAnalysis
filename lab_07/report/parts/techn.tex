\chapter{Технологическая часть}

В данном разделе описаны средства реализации, приведены листинги кода
и данные, на которых будет проводиться
тестирование.

\section{Средства реализации}

В данной работе для реализации был выбран язык программирования $Python$~\cite{Python}. 
Выбор обусловлен наличием опыта работы с ним. 

Замеры времени будут происходить с помощью функции \textit{process\_time} из библиотеки $time$~\cite{PC}. 

\section{Реализация алгоритмов}

В листинге \ref{lst:stand} представлен алгоритм стандартного поиска подстроки в строке, а в
листингах \ref{lst:mur} -- алгоритм Бойера-Мура и дополнительные к нему функции.

\noindent
\begin{lstinputlisting}[
	caption={Стандартный алгоритм поиска подстроки в строки},
	label={lst:stand},
	linerange={1-26}
]{../main/main.py}
\end{lstinputlisting}

\noindent
\begin{lstinputlisting}[
	caption={Алгоритм Бойера-Мура для поиска подстроки в строке},
	label={lst:mur},
	linerange={27-106}
]{../main/main.py}
\end{lstinputlisting}

\section{Функциональные тесты}
В таблице \ref{tbl:test} представлены функциональные тесты для стандартного алгоритма и алгоритма Бойера-Мура.

Входными данными являются строка и подстрока. Выходными данными является индекс, с которого начинается подстрока в строке.
\begin{table}[h]
	\begin{center}
		\caption{\label{tbl:test} Функциональные тесты}
		\begin{tabular}{|c|c|c|c|}
			\hline
			Входная строка & Входная подстрока & Результат & Ожидаемый результат\\ 
			\hline
			data & data  & 0 & 0\\ \hline
			bombambum & bem & -1 & -1\\ \hline
			aaaaaaaa  & bbbb  & -1 & -1\\ \hline
			a &  a & 0 & 0\\ \hline
			test data  & ta & 6 & 6\\
			\hline
		\end{tabular}
	\end{center}
\end{table}

Алгоритмы, реализованные в данной лабораторной работе, функциональные тесты прошли успешно.

\section*{Вывод}

В данном разделе были реализованы стандартный алгоритм поиска подстроки в строке и алгоритм Бойера-Мура. 
Для проверки правильности работы программы были написаны функциональные тесты.
