\chapter*{Заключение}
\addcontentsline{toc}{chapter}{Заключение}

В ходе исследования был проведен сравнительный анализ реализаций стандартного алгоритма и
алгоритма Бойера-Мура, по результатам которого можно сделать вывод, что для реализации
стандартного алгоритма наихудшим сценарием с точки зрения количества 
сравнений является ситуация, когда искомая подстрока находится в конце 
строки, а с учетом затрат времени, худшим случаем является отсутствие подстроки 
в строке. В отношении реализации алгоритма Бойера-Мура наихудшим сценарием как по 
количеству сравнений, так и по временным затратам является ситуация, когда 
искомая подстрока отсутствует в строке. 

Цель данной лабораторной работы, которая заключается в исследовании алгоритмов поиска подстроки в строке выполнена, 
также были выполнена следующие задачи:
\begin{itemize}
    \item описаны стандартный алгоритм и алгоритм Бойера-Мура для поиска подстроки в строке;
    \item разработано программное обеспечение, реализующее описанные алгоритмы;
    \item выполнены замеры процессорного времени работы алгоритма;
	\item проведен сравнительный анализ по времени работы реализаций алгоритма.
\end{itemize}
