\chapter*{Введение}
\addcontentsline{toc}{chapter}{Введение}

В программах, предназначенных для редактирования текста, часто необходимо найти 
все фрагменты текста, совпадающие с заданным образцом. Обычно, текст --- это редактируемый документ, 
а образец --- искомое слово, введённое пользователем.

Целью данной лабораторной работы является исследование алгоритмов поиска подстроки в строке.

Для достижения поставленной цели необходимо выполнить следующие задачи:
\begin{itemize}
    \item описать стандратный алгоритм и алгоритм Бойера-Мура для поиска подстроки в строке;
    \item разработать программное обеспечение, реализующее описанные алгоритмы;
    \item выполнить замеры процессорного времени работы алгоритмов;
	\item провести сравнительный анализ по времени работы реализаций алгоритмов.
\end{itemize}
