\section*{Введение}
\addcontentsline{toc}{chapter}{Введение}

\textbf{Коммивояжер} --- задача поиска минимального по стоимости маршрута по всем вершинам без 
повторений на полном взвешанном графе с n вершинами. Вершины графа являются городами, 
которые должен посетить коммивояжер, а веса ребер отражают расстояние(длины) или стоимость проезда.
Эта задача является NP-трудной, и точный переборный алгоритм ее решения имеет 
факториальную сложность~\cite{Ulianov}.

Для оптимизации поиска минимального маршрута используют муравьиный алгоритм, который базируется
на моделировании поведения колонии муравьев~\cite{Shtovba}.

\textbf{Целью данной лабораторной работы} является исследование метода 
решения задачи коммивояжера на базе муравьиного алгоритма.

Для достижения поставленной цели необходимо выполнить следующие задачи:
\begin{itemize}
    \item описать задачу коммивояжера;
    \item описать алгоритмы решения задачи коммивояжера -- полный перебор и муравьиный алгоритм;
    \item разработать программное обеспечение, реализующее данные алгоритмы;
    \item измерить процессорное время работы данных алгоритмов;
    \item провести сравнительный анализ по времени реализованных алгоритмов.
\end{itemize}
