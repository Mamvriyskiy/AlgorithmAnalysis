\chapter{Аналитическая часть}

В этом разделе будет представлена информация о задаче коммивояжера, 
а также о путях ее решения -- полным перебором и муравьиным алгоритмом.

\section{Задача коммивояжера}

\textbf{Задача коммивояжера}
заключается в поиске кратчайшего замкнутого маршрута, проходящего через все города
ровно 1 раз~\cite{Shtovba}. Если говорить формально, то необходимо найти
гамильтонов цикл минимального веса во взвешенном полном графе.

\section{Алгоритм полного перебора}

\textbf{Алгоритм полного перебора} для решения задачи коммивояжера предполагает
рассмотрение всех возможных путей в графе и выбор наименьшего из них. Данный
алгоритм имеет высокую сложность $O(n!)$, что влечет большие затраты по времени даже
при небольших значениях числа вершин в графе.

\section{Муравьиный алгоритм}

\textbf{Муравьиный алгоритм} основан на принципе поведения колонии муравьев.

Муравьи действуют, ощущая некий феромон. Каждый муравей, чтобы другие могли ориентироваться,
оставляет его на своем пути. При этом феромон испаряется для исключения случая, 
когда все муравьи движутся одним и тем же субоптимальным маршрутом. В результате при 
прохождении каждым муравьем различного маршрута наибольшее число феромона остается на оптимальном
пути.

Суть в том, что отдельно взятый муравей мало, что может, поскольку способен выполнять 
только максимально простые задачи. Но при большом количестве муравьи могут
самоорганизовываться в большие очереди для решения сложных задач.

Для каждого муравья переход из города $i$ в город $j$ зависит от трех
составляющих:
\begin{itemize}
    \item \textbf{память муравья} —-- список посещенных муравьем городов,
        заходить в которые еще раз нельзя. Используя этот список, муравей
        гарантированно не попадет в один и тот же город дважды.
    \item \textbf{видимость} —-- величина, обратная расстоянию: $\eta_{ij} = 1 / D_{ij}$,
        где $D_{ij}$ — расстояние между городами $i$ и $j$. Эта величина
        выражает эвристическое желание муравья поседить город $i$ из города
        $j$, чем ближе город, тем больше желание его посетить.
    \item \textbf{виртуальный след феромона} $\tau_{ij}$ на ребре $(i, j)$ представляет
        подтвержденное муравьиным опытом желание посетить город $j$ из города
        $i$.
\end{itemize}

Вероятность перехода $k$-ого муравья из города $i$ в город $j$ на $t$-й
итерации определяется формулой \ref{eq:1}:

\begin{equation}\label{eq:1}
      \begin{cases}
          P_{ij,k}(t) = \frac{[\tau_{ij}(t)]^{\alpha} \cdot
                        [\eta_{ij}]^{\beta}}{\displaystyle\sum_{l \in
                        J_{i,k}}[\tau_{il}(t)]^{\alpha} \cdot
                        [\eta_{il}]^{\beta}} & \quad \text{если } j \in
                        J_{i,k},\\
          P_{ij,k}(t) = 0 & \quad \text{если } j \notin J_{i,k}
      \end{cases}
      ,
\end{equation}

где $\alpha, \beta$ -- настраиваемые параметры, $J_{i,k}$ - список городов,
которые надо посетить $k$-ому муравью, находящемуся в $i$-ом городе, $\tau$ -
концентрация феромона, а при $\alpha = 0$ алгоритм вырождается в жадный.

После завершения движения всеми муравьями происходит обновление феромона.
Если $p \in [0, 1]$ -- коэффициент испарения феромона, то новое значения
феромона на ребре $(i,j)$ рассчитывается по формуле \ref{eq:2}:
\begin{equation}\label{eq:2}
    \tau_{ij}(t+1) = (1-p)\tau_{ij}(t) + \Delta \tau_{ij},~~\Delta \tau_{ij} =
                     \displaystyle\sum_{k=1}^N \tau^k_{ij}
\end{equation}
где
\begin{equation}\label{eq:3}
    \Delta \tau^k_{ij} = \begin{cases}
        \frac{Q}{L_k}, & \quad \textrm{ребро посещено k-ым муравьем,} \\
        0, & \quad \textrm{иначе}
    \end{cases}
\end{equation}
$L_{k}$ — длина пути k-ого муравья, $Q$ — некоторая константа порядка длины
путей, $N$ — количество муравьев ~\cite{Shtovba}.

\section*{Вывод}

В данном разделе была рассмотрена задача коммивояжера, а также алгоритм полного
перебор для ее решения и муравьиный алгоритм.
