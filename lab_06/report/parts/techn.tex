\chapter{Технологическая часть}

В данном разделе описаны требования к программному обеспечению, средства
реализации, приведены листинги кода и данные, на которых будет проводиться
тестирование.

\section{Требования к программному обеспечению}

Программа должна предоставлять следующие возможности:
\begin{itemize}[left=\parindent]
    \item ввод имени файла, хранящего матрицу смежности графа;
    \item ввод коэффициентов для муравьиного алгоритма;
    \item ввод максимального количества топлива;
    \item вывод искомого пути и минимальной затраты;
    \item подсчет времени работы алгоритмов на различных значениях количества
        вершин;
    \item параметризация муравьиного алоритма с оценкой точности при заданных
        параметрах.
\end{itemize}

\section{Средства реализации}
В данной работе для реализации был выбран язык программирования $Python \cite{python-lang}$, 
так как требуется замерить процессорное время для выполняемой программы, 
а также построить графики. Все эти инструменты присутствуют в выбранном языке программирования.

Время работы было замерено с помощью функции \textit{process\_time(...)} из библиотеки 
$time~\cite{python-lang-time}$.

\clearpage
\section{Реализация алгоритмов}

В листинге \ref{lst:full_comb} представлен алгоритм полного перебора путей, а в листингах \ref{lst:ant_alg}-\ref{lst:pher} -- муравьиный алгоритм и дополнительные к нему функции.

\begin{lstlisting}[label=lst:full_comb,caption=Релизация алгоритма полного пербора]
    def full_combinations(matrix, size, max_fuel):
    places = np.arange(size)
    places_combs = list()

    for combination in it.permutations(places):
        comb_arr = list(combination)
        places_combs.append(comb_arr)

    min_dist = float("inf")
    max_visit_cites = 0

    for i in range(len(places_combs)):
        flag = 0
        cur_dist = 0
        cur_visit = 0

        for j in range(size - 1):
            start_city = places_combs[i][j]
            end_city = places_combs[i][j + 1]

            cur_dist += matrix[start_city][end_city]
            if cur_dist > max_fuel:
                flag = 1
                cur_dist -= matrix[start_city][end_city]
                break

            cur_visit += 1

        if (cur_dist < min_dist and flag == 0):
            min_dist = cur_dist
            best_way = places_combs[i]
            max_visit_cites = cur_visit
        elif (max_visit_cites < cur_visit and cur_dist < min_dist):
            min_dist = cur_dist
            max_visit_cites = cur_visit
            best_way = places_combs[i][ : cur_visit]

    return min_dist, best_way
\end{lstlisting}

\begin{lstlisting}[label=lst:ant_alg,caption=Релизация муравьиного алгоритма]
    def ant_algorithm(matrix, places, alpha, beta, k_evaporation, days, max_fuel):
    q = calc_q(matrix, places)

    best_way = []
    min_dist = float("inf")
    max_visit_cites = 0

    pheromones = get_pheromones(places)
    visibility = get_visibility(matrix, places)

    ants = places

    for _ in range(days):
        route = np.arange(places)
        visited = get_visited_places(route, ants)

        for ant in range(ants):
            while len(visited[ant]) != places:
                pk = find_posibilities(pheromones, visibility, visited, places, ant, alpha, beta)
                chosen_place = choose_next_place_by_posibility(pk)
                visited[ant].append(chosen_place - 1)


            cur_dist, cur_visit, flag  = res_calc_length(matrix, visited[ant], max_fuel)

            if (cur_dist < min_dist and flag == 0):
                min_dist = cur_dist
                best_way = visited[ant]
                max_visit_cites = cur_visit
            elif (max_visit_cites < cur_visit and cur_dist < min_dist):
                min_dist = cur_dist
                max_visit_cites = cur_visit
                best_way = visited[ant][ : cur_visit]

        pheromones = update_pheromones(matrix, places, visited, pheromones, q, k_evaporation)

    return min_dist, best_way
\end{lstlisting}

\noindent
\begin{minipage}{\linewidth}
\begin{lstinputlisting}[
	caption={Алгоритм нахождения массива вероятностных переходов в непосещенные города},
	label={lst:prob},
	linerange={59-76}
]{../main/ant.py}
\end{lstinputlisting}
\end{minipage}

\noindent
\begin{minipage}{\linewidth}
\begin{lstinputlisting}[
	caption={Алгоритм нахождения следующего города на основании рандома},
	label={lst:rand},
	linerange={78-86}
]{../main/ant.py}
\end{lstinputlisting}
\end{minipage}

\noindent
\begin{minipage}{\linewidth}
\begin{lstinputlisting}[
	caption={Алгоритм обновления матрицы феромонов},
	label={lst:pher},
	linerange={40-57}
]{../main/ant.py}
\end{lstinputlisting}
\end{minipage}

\clearpage

\section{Функциональные тесты}
В таблице \ref{tab:test} представлены функциональные тесты для алгоритма полного перебора
и муравьиного алгоритма.
Входными данными являются матрица смежностей и количество топлива, данное для осуществления маршрута. 
Выходными данными являются маршрут, который удалось пройти с минимальными затратами, и количество топлива, потраченного на данный маршрут.
\begin{table}[h!]
	\begin{center}
        \caption{\label{tab:test}Функциональные тесты}
        \begin{tabular}{|c|c|c|}
			\hline
            \textbf{Входные данные} & \textbf{Вывод} &\textbf{Ожидаемый
            результат} \\ [2mm]
            \hline
            $ \begin{pmatrix}
                0 &  4 &  2 &  1 & 7 \\
                4 &  0 &  3 &  7 & 2 \\
                2 &  3 &  0 & 10 & 3 \\
                1 &  7 & 10 &  0 & 9 \\
                7 &  2 &  3 &  9 & 0
            \end{pmatrix}$, 16 &
                8, [0, 2, 4, 1, 3] &
                8, [0, 2, 4, 1, 3] \\

            $ \begin{pmatrix}
                    0 & 1 & 2 \\
                    1 & 0 & 1 \\
                    2 & 1 & 0	
            \end{pmatrix}$, 10 &
                4, [0, 1, 2] &
                4, [0, 1, 2] \\

            $ \begin{pmatrix}
                0 & 15 & 19 & 20 \\
                15 &  0 & 12 & 13 \\
                19 & 12 &  0 & 17 \\
                20 & 13 & 17 &  0
            \end{pmatrix}$, 30 &
                27, [0, 1, 3] &
                27, [0, 1, 3] \\

            $ \begin{pmatrix}
                0 & 15 & 19 & 20 \\
                15 &  0 & 12 & 13 \\
                19 & 12 &  0 & 17 \\
                20 & 13 & 17 &  0
            \end{pmatrix}$, 50 &
                44, [0, 1, 2, 3] &
                44, [0, 1, 2, 3] \\
            \hline 
		\end{tabular}
	\end{center}
\end{table}

Алгоритмы, реализованные в данной лабораторной работе, функциональные тесты прошли успешно.


\section*{Вывод}

В данном разделе были реализованы алгоритмы решения задачи коммивояжера: полный
перебор и муравьиный алгоритм. Также были написаны функциональные тесты для проверки правильности 
работы программы.

