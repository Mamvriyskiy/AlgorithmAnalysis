\chapter*{Заключение}
\addcontentsline{toc}{chapter}{Заключение}

В результате исследования было определено, 
что при малых значениях количества вершин, до 8, следует использовать алгоримт полного перебора, так 
как он в любом случае дает точный результат. При количестве вершин больше 8 следует использовать муравьиный алгоритм, 
так как алгоритм полного перебора при таких значениях работает неопределенно долго.

При использовании муравьиного алгоритма необходимо выбирать оптимальные по результатам параметризации
значения параметров.

Цель, которая заключается в исследовании метода 
решения задачи коммивояжера на базе муравьиного алгоритма, 
выполнена, также в ходе выполнения лабораторной 
работы были решены следующие задачи:
\begin{itemize}
    \item описана задача коммивояжера;
    \item описаны алгоритмы решения задачи коммивояжера --- полный перебор и муравьиный алгоритм;
    \item разработано программное обеспечение, реализующее данные алгоритмы;
    \item измеренно процессорное время работы данных алгоритмов;
    \item проведен сравнительный анализ по времени реализованных алгоритмов.
\end{itemize}
