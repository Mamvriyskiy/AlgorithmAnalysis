\chapter*{Введение}
\addcontentsline{toc}{chapter}{Введение}

Сегодня компьютерам требуется производить все более трудоемкие вычисления на больших
объемах данных. При этом предъявляются требования к скорости вычислений.

При обработке больших объемов данных часто возникает ситуация, когда каждых
набор данных необходимо обработать последовательно несколькими алгоритмами. Так
как каждый набор может быть обработан независимого от другого имеется
возможность распараллелить вычисления. В таком случае реализуют конвейерную
обработку данных, когда каждый алгоритм выполняется на отдельной ленте,
запущенной в отдельном потоке, а результат каждого этапа используется в
качестве входных данных для следующего.

\textbf{Целью данной работы} является исследование навыков
конвейерной обработки данных.

Для достижения поставленной цели необходимо выполнить следующие задачи:
\begin{itemize}[left=\parindent]
    \item описать основы конвейерной обработки данных;
    \item описать алгоритмы, реализуемые на каждом этапе конвейера;
    \item разработать описанные алгоритмы;
    \item разработать конвейерную обработку данных;
    \item разработать последовательную обработку данных;
    \item провести тестирование реализованных алгоритмов;
	\item выполнить замеры процессорного времени работы конвейерной и последовательной обработок;
	\item провести сравнительный анализ по времени работы двух видов обработок.
\end{itemize}
