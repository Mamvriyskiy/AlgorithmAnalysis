\chapter{Технологическая часть}

В данном разделе описаны требования к программному обеспечению, средства реализации, приведены листинги кода и данные, 
на которых будет проводиться тестирование.

\section{Средства реализации}
Для реализации данной лабораторной работы выбран компилируемый многопоточный
язык програмирования \texttt{Go}~\cite{golang}, так как он предоставляет
необходимый функционал для работы с потоками и простоую реализацию их
взаимодействия.

Замеры времени проводились при помощи getThreadCpuTimeNs функции, написанной на C, 
подключенной с помощью CGO~\cite{cgo}.

\newpage
\section{Реализация алгоритмов}

В данном подразделе представлены листинги кода алгоритмов:
\begin{itemize}
    \item реализация линейной обработки \ref{lst:line};
    \item реализация конвейерной обработки \ref{lst:pip};
    \item реализация стандартного алгоритма умножения матриц \ref{lst:stand}
    \item реализация алгоритма Винограда умножения матриц \ref{lst:vino}
\end{itemize}

\begin{lstlisting}[label=lst:line,caption=Релизация линейной обработки]
func serialMtr(sliceMtr []matrixStruct, count int) {
    for i := range sliceMtr {
        sliceMtr[i].inverseMatrix = gaussMethodEx(sliceMtr[i].matrixA)
        sliceMtr[i].standartMul = mulStandart(sliceMtr[i].inverseMatrix, sliceMtr[i].matrixB, sliceMtr[i].size)
        sliceMtr[i].vinogradMul = mulVinogradov(sliceMtr[i].inverseMatrix, sliceMtr[i].matrixB, sliceMtr[i].size)
    }
}
\end{lstlisting}

\begin{lstinputlisting}[
    caption={Релизация линейной обработки},
    label={lst:pip},
    linerange={8-82}
]{../main/pipeline.go}
\end{lstinputlisting}

\clearpage
\noindent
\begin{lstinputlisting}[
	caption={Реализация алгоритма стандартного метода умножения матриц},
	label={lst:stand},
	linerange={3-16}
]{../main/standart.go}
\end{lstinputlisting}

\noindent
\begin{lstinputlisting}[
	caption={Реализация алгоритма Винограда умножения матриц},
	label={lst:vino},
	linerange={3-48}
]{../main/vinograd.go}
\end{lstinputlisting}

\section{Функциональные тесты}

В таблице \ref{tab:test} представлены функциональные тесты для алгоритмов умножения матриц, а в таблице 
\ref{tab:tests} приведены функциональные тесты для алгоритма
нахождения обратной матрицы.

\clearpage
\begin{table}[h!]
	\begin{center}
        \caption{\label{tab:test}Функциональные тесты}
        \begin{tabular}{|c|c|c|}
			\hline
            \textbf{Матрица 1} & \textbf{Матрица 2} &\textbf{Ожидаемый
            результат} \\ [2mm]
            \hline
			$\begin{pmatrix}
                1 & 2 & 3\\
                1 & 2 & 3\\
                1 & 2 & 3
			\end{pmatrix}$ &
			$\begin{pmatrix}
                -1 & 2 & 3\\
                1 & -2 & 3\\
                1 & 2 & -3
			\end{pmatrix}$ &
			$\begin{pmatrix}
                4 & 4 & 0\\
                4 & 4 & 0\\
                4 & 4 & 0
			\end{pmatrix}$ \\
            \hline
			$\begin{pmatrix}
                1 & 2 & 3\\
                1 & 2 & 3
			\end{pmatrix}$ &
			$\begin{pmatrix}
                -1 & -3\\
                -2 & -2\\
                -3 & -1
			\end{pmatrix}$ &
			$\begin{pmatrix}
                -14 & -10\\
                -14 & -10
			\end{pmatrix}$ \\
            \hline
			$\begin{pmatrix}
                3 & -5\\
                1 & -2
			\end{pmatrix}$ &
			$\begin{pmatrix}
                2 & -5\\
                1 & -3
			\end{pmatrix}$ &
			$\begin{pmatrix}
			    1 & 0\\
                0 & 1
			\end{pmatrix}$ \\
            \hline
			$\begin{pmatrix}
                1 & 2 & 3\\
                1 & 2 & 3\\
                1 & 2 & 3
			\end{pmatrix}$ &
			$\begin{pmatrix}
                0 & 0 & 0\\
                0 & 0 & 0\\
                0 & 0 & 0
			\end{pmatrix}$ &
			$\begin{pmatrix}
                0 & 0 & 0\\
                0 & 0 & 0\\
                0 & 0 & 0
			\end{pmatrix}$ \\
            \hline
			$\begin{pmatrix}
                1 & 2 & 3\\
                1 & 2 & 3\\
                1 & 2 & 3
			\end{pmatrix}$ &
			$\begin{pmatrix}
                1 & 0 & 0\\
                0 & 1 & 0\\
                0 & 0 & 1
			\end{pmatrix}$ &
			$\begin{pmatrix}
                1 & 2 & 3\\
                1 & 2 & 3\\
                1 & 2 & 3
            \end{pmatrix}$ \\
            \hline
		\end{tabular}
	\end{center}
\end{table}

\begin{table}[h!]
	\begin{center}
        \caption{\label{tab:tests}Функциональные тесты}
        \begin{tabular}{|c|c|}
			\hline
            \textbf{Матрица} & \textbf{Ожидаемый
            результат}\\ [2mm]
            \hline
			$\begin{pmatrix}
                5 & 9 & 2\\
                6 & 11 & 4\\
                1 & 3 & 1
			\end{pmatrix}$ &
			$\begin{pmatrix}
                0.1 & 0.3 & -2\\
                0.2 & 0 & 0.9\\
                -1 & 0.7 & 0
			\end{pmatrix}$\\
            \hline
            $\begin{pmatrix}
                1 & 0 & 0 \\
                0 & 2 & 0 \\
                0 & 0 & 3 \\
			\end{pmatrix}$ &
			$\begin{pmatrix}
                1 & 0 & 0 \\
                0 & 0.5 & 0 \\
                0 & 0& 0.33333 \\
			\end{pmatrix}$\\
            \hline
			$\begin{pmatrix}
                2 & 1 & -1 \\
                3 & 2 & -2 \\
                1 & -1 & 2
			\end{pmatrix}$ &
			$\begin{pmatrix}
                2 & -1 & 0 \\
                -8 & 5 & 1 \\
                -5 & 3 & 1
			\end{pmatrix}$\\
            \hline
			$\begin{pmatrix}
                1 & 2\\
                3 & 4\\
			\end{pmatrix}$ &
            $\begin{pmatrix}
                -2 & 1\\
                1.5 & -0.5
			\end{pmatrix}$\\
            \hline
		\end{tabular}
	\end{center}
\end{table}

Алгоритмы, реализованные в данной лабораторной работе, функциональные тесты прошли успешно.

\clearpage
\section*{Вывод}

В данном разделе были реализованы последовательный и конвейрный алгоритмы
поэтапной обрабокти матриц. Также были написаны функциональные тесты для проверки
правильности работы программы.
