\chapter*{Введение}
\addcontentsline{toc}{chapter}{Введение}

Сортировка -- процесс перегруппировки заданной последовательности объектов в
некотором определеннном порядке.

Целью алгоритмов сортировки является упорядочение последовательности элементов данных. 
Существует множество различных методов сортировки данных. 
Однако любой алгоритм сортировки можно разбить на три основные части:

\begin{itemize}
	\item сравнение, определяющее упорядочность пары элементов;
	\item перестановка, меняющая местами пару элементов;
	\item собственно сортирующий алгоритм, который осуществляет сравнение
	и перестановку элементов данных до тех пор, пока все эти элементы
	не будут упорядочены.
\end{itemize}

Целью данной лабораторной работы является исследование алгоритмов сортировок.

\leavevmode\newline
Для достижения поставленной цели небходимо выполнить следующие задачи:
\begin{itemize}
	\item Описать три алгоритма сортировки: блинная, пирамидальная, перемешиванием;
	\item Создать программное обеспечение, реализующее алгоритмы сортировки, указанные в варианте;
	\item Оценить трудоемкости сортировок;
	\item Провести анализ затрат работы программы по времени, выяснить влияющие на них характристики.
\end{itemize}
