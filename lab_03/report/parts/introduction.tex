\chapter*{Введение}
\addcontentsline{toc}{chapter}{Введение}

Одной из важнейших процедур обработки структурированной информации является сортировка.

Сортировка - процесс перегруппировки заданной последовательности (кортежа) объектов в
некотором определеннном порядке. Такой опредленный порядок позволяет, в некоторых случаях,
эффективнее и удобнее работать с заданной последовательностью. В частности, одной из
целей сортировки является облегчение задачи поиска элемента в отсортированном множестве. 

Алгоритмы сортировки используются практически в любой программой системе. 
Целью алгоритмов сортирови является упорядочение последовательности элементов данных. 
Поиск элемента в последовательности отсортированных данных занимает время, пропорциональное
логарифму количеству элементов в последовательности, а поиск элемента в последовательности
не отсортированных данных занимает время, пропорциональное воличеству элементов в
последовательности, то есть намного больше. Существует множество различных методов 
сортировки данных. Однако любой алгоритм сортировки можно разбить на три основные части:

\begin{itemize}
	\item сравнение, определяющее упорядочность пары элементов;
	\item перестановка, меняющая местами пару элементов;
	\item собственно сортирующий алгоритм, который осуществляет сравнение
	и перестановку элементов данных до тех пор, пока все эти элементы
	не будут упорядочены.
\end{itemize}

Одной из важнейшей характеристикой любого алгоритма сортировки является скорость 
его работы, которая определяется функциональной зависимостью среднего времени сортировки
последовательностей элементов данных, определенной длины, от этой длины.


\leavevmode\newline
Задачи данной лабораторной:
\begin{itemize}
	\item изучить и реализовать три алгоритма сортировки: блинная, пирамидальная, перемешиванием;
	\item Выбрать инструменты для замера процессорного времени выполнения реализаций алгоритмов;
	\item Создать ПО, реализующее алгоритмы сортировки, указанные в варианте;
	\item Провести анализ затрат работы программы по времени и по памяти, выяснить влияющие на них характристики;
\end{itemize}
