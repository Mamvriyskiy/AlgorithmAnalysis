\chapter*{Заключение}

В ходе исследования был проведен сравнительный анализ алгоритмов, в результате
которого было выяснено, что пирамидальная сортировка при рандомном расположении элементов в срезе 
работает быстрее сортировки перемешиванием примерно в 4 -- 30 раз при длине среза от 10 -- 600 и в 53 -- 225 раз быстрее при длине среза 700 -- 1000.
Данное превосходство резко уменьшается при отсортированной последовательности, выигрыш по времени при таком расположении элементов
в 3 -- 30 раз.
Если сравнивать блинную сортировку и сортировку перемешиванием, то вторая работает быстрее 
примерно в 1 -- 3 раза независимо от расположения элементов. 

Из результатов оценки трудоемкости следует, что пирамидальная сортировка имеет наименьшую сложность, 
которая составляет $O(N \cdot logN)$, далее следует сортировка перемешиванием, которая при лучшем случае имеет 
сложность O(N), а при худшем случае O($N^2$). Наибольшую трудоемкость имеет блинная сортировка, у которой сложность O($N^2$).
Лучший случай данной сортировки имеет примерно такую же трудоемксоть, как худший случай сортировки перемешиванием, а худший случай 
примерно в 4 раза имеет большую трудоемкость, чем худший случай сортировки перемешиванием.

Исходя из данных показателей пирамидальная сортировка работает быстрее всех,
а блинная дольше всех независимо от расположения элементов в последовательности.

Цель данной лабораторной работы, которая заключается в исследовании алгоритмов сортировок, выполнена. Также были выполнены следующие задачи:

\begin{itemize}[left=\parindent]
    \item описаны алгоритмы пирамидальной, блинной сортировки и 
        сортировки перемешиванием;
    \item cоздано программное обеспечение, реализующее алгоритмы данных сортировок;
    \item произведена оценка трудоемкости каждого из алгоритмов;
    \item по экспериментальным данным сделаны выводы об эффективности по
        времени каждого из реализованных алгоритмов, которые были подтверждены
        теоретическими расчетами трудоемкости алгоритмов.
\end{itemize}
