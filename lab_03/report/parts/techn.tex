\chapter{Технологическая часть}

В данном разделе описаны требования к программному обеспечению, средства
реализации, приведены листинги кода и данные, на которых будет проводиться
тестирование.


\section{Средства реализации}

В качестве языка программирования для реализации данной лабораторной работы был выбран язык GO\cite{go}. Данный
выбор обусловлен тем, что данный язык является типизированным, имеет инструменты 
для тестирования, профилирования и форматирования кода.

Замеры времени проводились при помощи getThreadCpuTimeNs функции, написанной на C, 
подключенной с помощью CGO.\cite{cgo}

\section{Реализация алгоритмов}

В данном подразделе представлены листинги кода ранее описанных алгоритмов:
\begin{itemize}[left=\parindent]
    \item алгоритм пирамидальной сортировки (листинг \ref{lst:heapSort});
    \item алгоритм блинной сортировки (листинги \ref{lst:pancakeSort});
    \item алгоритм сортировки перемешивания (листинги \ref{lst:shaker}).
\end{itemize}

\noindent
\begin{minipage}{\linewidth}
\begin{lstinputlisting}[
	caption={Реализация алгоритма пирамидальной сортировки.},
	label={lst:heapSort},
	linerange={3-35}
]{../main/sorted/heapSort.go}
\end{lstinputlisting}
\end{minipage}

\noindent
\begin{minipage}{\linewidth}
\begin{lstinputlisting}[
	caption={Реализация алгоритма блинной сортировки.},
	label={lst:pancakeSort},
	linerange={3-41}
]{../main/sorted/pancakeSort.go}
\end{lstinputlisting}
\end{minipage}

\noindent
\begin{minipage}{\linewidth}
\begin{lstinputlisting}[
	caption={Реализация алгоритма сортировки перемешивания.},
	label={lst:shaker},
	linerange={3-20}
]{../main/sorted/shakerSort.go}
\end{lstinputlisting}
\end{minipage}

\section{Функциональные тесты}

В таблице \ref{tbl:functional_test} приведены тесты для функций, реализующих алгоритмы сортировки.

\begin{table}[h]
	\begin{center}
		\caption{\label{tbl:functional_test} Функциональные тесты.}
		\begin{tabular}{|c|c|c|}
			\hline
			Входной массив & Ожидаемый результат & Результат \\ 
			\hline
			$[1,2,3,4]$ & $[1,2,3,4]$  & $[1,2,3,4]$\\
			$[5,4,3,2,1]$  & $[1,2,3,4,5]$ & $[1,2,3,4,5]$\\
			$[3,2,-5,0,1]$  & $[-5,0,0,2,3]$  & $[-5,0,0,2,3]$\\
			$[1]$  & $[1]$  & $[1]$\\
			$[]$  & $[]$  & $[]$\\
			\hline
		\end{tabular}
	\end{center}
\end{table}

Алгоритмы, реализованные в данной лабораторной работе, функци-
ональные тесты прошли успешно.

\section*{Вывод}

Были разработаны схемы всех трех алгоритмов сортировки.
Для дальнейшей проверки правильности работы программы были выделены
классы эквивалентности тестов.

