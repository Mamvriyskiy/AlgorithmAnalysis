\chapter*{Заключение}
\addcontentsline{toc}{chapter}{Заключение}

В результате исследования было определено, 
что стандартный алгоритм умножения матриц проигрывает по времени алгоритму Винограда
примерно в 1.2 раза из-за того, что в алгоримте Винограда часть вычислений происходит
заранее, а также сокращается часть сложных операций -
операций, умножения, поэтому предпочтение следует отдавать алгоритму Винограда. Но 
наилучшие показатели как по времени, так и по трудоемкости демонстрирует оптимизированный алгоритм
Винограда. Он работает примерно 
в 1.4 раза быстрее и имеет трудоемкость, меньшую примерно в 1.25 раза, благодаря замене операции равно и плюс 
на операцию плюс-равно, введению буфера, а также замене операции умножения на сдвиг. Поэтому при выборе
самого быстрого алгоритма предпочтение стоит отдавать оптимизированному алгоритму Винограда.

Однако по памяти алгоритм Винограда проигрывает стандартному алгоритму, так как 
происходит выделение памяти под хранение дополнительных данных, полученных при предварительной обработке.

Цель, которая заключается в исследовании стандартного алгоритма умножения матриц и 
алгоритма Винограда, а также его оптимизация, выполнена, также в ходе выполнения лабораторной 
работы были решены следующие задачи:
\begin{itemize}[left=\parindent]
    \item были описаны алгоритмы умножения матриц: стандартный и Винограда;
    \item создано программное обеспечение, реализующее следующие алгоритмы:
	\begin{itemize}[label=---]
		\item классический алгоритм умножения матриц;
		\item алгоритм Винограда;
		\item оптимизированный алгоритм Винограда.
	\end{itemize}
    \item была произведена оценка трудоемкости каждого из алгоритмов;
    \item по экспериментальным данным были сделаны выводы об эффективности по 
    времени и памяти реализованных алгоритмов.
\end{itemize}
