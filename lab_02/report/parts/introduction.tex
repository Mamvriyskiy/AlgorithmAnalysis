\chapter*{Введение}
\addcontentsline{toc}{chapter}{Введение}

Умножение матриц имеет многочисленное применение в физике,
математике, программировании. При этом сложность стандартного алгоритма умножения матриц
N x N составляет $O(N^3)$, что послужило причиной разработки новых алгоритмов меньшей сложности. 
Одним из таких алгоритмов является алгоритм Винограда со сложностью $(O(N^{2.3755}))$.

Целью данной лабораторной работы является исследование стандартного алгоритма умножения матриц и 
алгоритма Винограда, а также его оптимизация.

Для достижения поставленнной цели необходимо выполнить следующие задачи:

\begin{enumerate}[label={\arabic*)}]
	\item Проанализировать данные алгоритмы умножения матриц;
	\item Создать программное обеспечение, реализующее следующие алгоритмы:
	\begin{itemize}[label=---]
		\item классический алгоритм умножения матриц;
		\item алгоритм Винограда;
		\item оптимизированный алгоритм Винограда.
	\end{itemize}
	\item Оценить трудоемкости алгоритмов умножения матриц;
	\item Провести анализ затрат работы программы по процессорному времени;
	\item Провести сравнительный анализ между алгоритмами.
\end{enumerate}
