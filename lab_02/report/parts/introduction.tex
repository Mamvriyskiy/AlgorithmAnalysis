\chapter*{Введение}

Умножение матриц является имеет многочисленное применение в физике,м
математике, программировании. При этом сложность стандартного алгоритма умножения матриц
N x N составляет $O(N^3)$, что послужило причиной разработки новых алгоритмов меньшей сложности. 
Одним из таких алгоритмов является алгоритм Виноградова со сложностью $(O(N^{2.3755}))$.

Целью данной лабораторной работы является изучение алгоритмов умножения матриц
таких, как: стандартный и Виногрдова. Также будут получены навыки расчета сложности алгоритмов 
и их оптимизация.

Для достижения поставленнной цели необходимо выполнить следующие задачи:

\begin{enumerate}[label={\arabic*)}]
	\item Описать данные алгоритмы умножения матриц;
	\item Создать программное обеспечение, реализующее следующие алгоритмы:
	\begin{itemize}[label=---]
		\item классический алгоритм умножения матриц;
		\item алгоритм Винограда;
		\item оптимизированный алгоритм Винограда.
	\end{itemize}
\end{enumerate}

% \begin{enumerate}[label={\arabic*)}]
% 	\item Описать данные алгоритмы умножения матриц;
% 	\item Создать программное обеспечение, реализующее следующие алгоритмы:
% 	\begin{itemize}[label=---]
% 		\item классический алгоритм умножения матриц;
% 		\item алгоритм Винограда;
% 		\item оптимизированный алгоритм Винограда.
% 	\end{itemize}
% 	\item Оценить трудоемкости сортировок;
% 	\item Провести анализ затрат работы программы по процессорному времени;
% 	\item Провести сравнительный анализ между алгоритмами.
% \end{enumerate}
