\chapter{Аналитическая часть}

В данном разделе будут рассмотрены стандартный алгоритм 
умножения матриц, алгоритм Винограда и его оптимизация.

\section{Стандартный алгоритм умножения матриц}

Стандартный алгоритм умножения матриц является реализацией математического определения
произведения матриц, которое формулируется следующим образом: пусть даны две 

\begin{equation}
    A_{np} = \begin{pmatrix}
            a_{11} & a_{12} & \ldots & a_{1m}\\
            a_{21} & a_{22} & \ldots & a_{2m}\\
            \vdots & \vdots & \ddots & \vdots\\
            a_{l1} & a_{l2} & \ldots & a_{lm}
    \end{pmatrix},
\end{equation}

имеющая n строк и p столбцов,

\begin{equation}
    B_{pm} = \begin{pmatrix}
            b_{11} & b_{12} & \ldots & b_{1n}\\
            b_{21} & b_{22} & \ldots & b_{2n}\\
            \vdots & \vdots & \ddots & \vdots\\
            b_{m1} & b_{m2} & \ldots & b_{mn}
        \end{pmatrix},
\end{equation}

имеющая p строк и m столбцов, тогда матрица $C$ будет иметь n строк и m столбцов
\begin{equation}
	C_{nm} = \begin{pmatrix}
		c_{11} & c_{12} & \ldots & c_{1n}\\
		c_{21} & c_{22} & \ldots & c_{2n}\\
		\vdots & \vdots & \ddots & \vdots\\
		c_{l1} & c_{l2} & \ldots & c_{ln}
	\end{pmatrix},
\end{equation}

где
\begin{equation}
	\label{eq:M}
	c_{ij} =
	\sum_{r=1}^{m} a_{ir}b_{rj} \quad (i=\overline{1,l}; j=\overline{1,n})
\end{equation}
- называется \textbf{произведением матриц} $A$ и $B$.

\clearpage

\section{Алгоритм Винограда для умножения матриц}
Из результата умножения двух матриц следует, что каждый элемент в нем представляет
собой скалярное произведение соответствующих строки и столбца исходных матриц.
Такое умножение допускает предварительную обработку,
позволяющую часть работы выполнить заранее.

Рассмотрим два вектора $V = (v_1, v_2, v_3, v_4)$ и $W = (w_1, w_2, w_3, w_4)$.
Их скалярное произведение равно: $V \cdot W = v_1w_1 + v_2w_2 + v_3w_3 + v_4w_4$, что эквивалентно (\ref{for:new}):
\begin{equation}
	\label{for:new}
		V \cdot W = (v_1 + w_2)(v_2 + w_1) + (v_3 + w_4)(v_4 + w_3) - v_1v_2 - v_3v_4 - w_1w_2 - w_3w_4.
\end{equation}

Несмотря на то, что второе выражение требует вычисление большего количества операций, чем стандартный алгоритм: вместо четырёх умножений -- шесть, а вместо трёх сложений -- десять, выражение в правой части последнего равенства допускает предварительную обработку: его части можно вычислить заранее и запомнить для каждой строки первой матрицы и для каждого столбца второй, что позволит для каждого элемента выполнять лишь два умножения и пять сложений, складывая затем только лишь с 2 предварительно посчитанными суммами соседних элементов текущих строк и столбцов.
Из-за того, что операция сложения быстрее операции умножения в ЭВМ, на практике алгоритм должен работать быстрее стандартного.\cite{winograd}

\clearpage
\section{Оптимизированный алгоритм Винограда}

При программной реализации рассмотренного выше алгоритма Винограда можно сделать 
следующие оптимизации:
\begin{enumerate}
	\item значение $\frac{N}{2}$, используемое как ограничение цикла подсчёта предварительных
        данных, можно кэшировать;
	\item операцию умножения на 2 программно эффективнее реализовывать как побитовый сдвиг 
        влево на 1;
	\item операции сложения и вычитания с присваиванием следует реализовывать при помощи 
        соответствующего оператора $+=$ или $-=$ (при наличии данных операторов в выбранном 
        языке программирования);
    \item вместо постоянного обращения к элементам матрицы по индексам, можно использовать буфер, 
        в который будет записан результат выполнения цикла и потом который будет присвоен
        элементу матрицы.
\end{enumerate}

\section{Вывод}
В данном разделе были рассмотрены алгоритмы умножения матриц – стандартного и Винограда, 
который имеет большую эффективность за счёт предварительной обработки, а также количество операций умножения. 
Были рассмотрены оптимизации, которые можно учесть при программной реализации данных
алгоритма Винограда.

\clearpage
