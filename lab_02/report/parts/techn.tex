\chapter{Технологическая часть}

В данном разделе описаны требования к программному обеспечению, средства
реализации, приведены листинги кодов и данные, на которых будет проводиться
тестирование.

\section{Средства реализации}

В качестве языка программирования для реализации данной лабораторной работы был выбран язык GO\cite{go}. Данный
выбор обусловлен тем, что данный язык является типизированным, имеет инструменты 
для тестирования, профилирования и форматирования кода, быстрая скорость компиляции.

Замеры времени проводились при помощи getThreadCpuTimeNs функции, написанной на C, 
подключенной с помощью CGO.\cite{cgo}

\section{Реализация алгоритмов}

В данном подразделе представлены листинги кода ранее описанных алгоритмов:
\begin{itemize}[left=\parindent]
    \item стандартный алгоритм умножения матриц (листинг \ref{lst:std});
    \item алгоритм Винограда (листинги \ref{lst:Winograd});
    \item оптимизированный алгоритм Винограда (листинги
        \ref{lst:WinogradOpt}).
\end{itemize}

\clearpage
\begin{lstlisting}[label=lst:std,caption=Реализация стандартного алгоритма умножения матриц]
func MulMAtrixA(matrixA, matrixB [][]int, n, m, p int) [][]int {
        matrixC := make([][]int, n)
        for i := 0; i < n; i++ {
            matrixC[i] = make([]int, m) 
            for j := 0; j < m; j++ { 
                for k := 0; k < p; k++ {
                    matrixC[i][j] = matrixC[i][j] + matrixA[i][k] * matrixB[k][j]
                }
            }
        }

        return matrixC
    }
\end{lstlisting}

\begin{lstlisting}[label=lst:Winograd,caption=Реализация алгоритма Винограда умножения матриц]
func MulVinogradovA(matrixA, matrixB [][]int, a, b, c int) [][]int {
    row := make([]int, a)
    column := make([]int, c)
    matrixC := make([][]int, a)

    for i := 0; i < a; i++ {
        matrixC[i] = make([]int, c)
        for j := 0; j < b / 2 ; j++ {
            row[i] = row[i] + matrixA[i][2 * j] * matrixA[i][2 * j + 1]
        }
    }

    for i := 0; i < c; i++ {
        for j := 0; j < b / 2; j++ {
            column[i] = column[i] + matrixB[2 * j][i]*matrixB[2 * j + 1][i] 
        }
    }

    for i := 0; i < a; i++ {
        for j := 0; j < c; j++ {
            matrixC[i][j] = -row[i] - column[j]
            for k := 0; k < b / 2; k++ {
                matrixC[i][j] = matrixC[i][j] + (matrixA[i][2 * k + 1] + matrixB[2 * k][j]) *
                    (matrixA[i][2 * k] + matrixB[2 * k + 1][j])
            }
        }
    }

    if (b % 2 == 1) {
        for i := 0; i < a; i++ {
            for j := 0; j < c; j++ {
                matrixC[i][j] = matrixC[i][j] + matrixA[i][b - 1] * matrixB[b - 1][j]
            }
        }
    }   

    return matrixC
}
\end{lstlisting}

\begin{lstlisting}[label=lst:WinogradOpt,caption=Реализация оптимизированного алгоритма Винограда умножения матриц]
func MulVinogradovB(matrixA, matrixB [][]int, a, b, c int) [][]int {
    row := make([]int, a)
    column := make([]int, c)
    matrixC := make([][]int, a)

    num := b / 2

    for i := 0; i < a; i++ {
        matrixC[i] = make([]int, c)
        buf := 0
        for j := 0; j < num ; j++ {
            buf += matrixA[i][j << 1] * matrixA[i][j << 1 + 1]
        }
        row[i] = buf
    }

    for i := 0; i < c; i++ {
        buf := 0
        for j := 0; j < num; j++ {
            buf += matrixB[j << 1][i] * matrixB[j << 1 + 1][i] 
        }
        column[i] = buf
    }

    for i := 0; i < a; i++ {
        for j := 0; j < c; j++ {
            buf := -row[i] - column[j]
            for k := 0; k < num; k++ {
                buf += (matrixA[i][k << 1 + 1] + matrixB[k << 1][j]) *
                    (matrixA[i][k << 1] + matrixB[k << 1 + 1][j])
            }
            matrixC[i][j] = buf
        }
    }

    if (b % 2 == 1) {
        l := b - 1
        for i := 0; i < a; i++ {
            for j := 0; j < c; j++ {
                matrixC[i][j] += matrixA[i][l] * matrixB[l][j]
            }
        }
    }   

    return matrixC
}
\end{lstlisting}

\clearpage
\section{Описание тестирования}

В таблице \ref{tab:test} представлены функциональные тесты для алгоритмов умножения матриц.
\begin{table}[h!]
	\begin{center}
        \caption{\label{tab:test}Функциональные тесты}
        \begin{tabular}{|c|c|c|}
			\hline
            \textbf{Матрица 1} & \textbf{Матрица 2} &\textbf{Ожидаемый
            результат} \\ [2mm]
            \hline
			$\begin{pmatrix}
                1 & 2 & 3\\
                1 & 2 & 3\\
                1 & 2 & 3
			\end{pmatrix}$ &
			$\begin{pmatrix}
                -1 & 2 & 3\\
                1 & -2 & 3\\
                1 & 2 & -3
			\end{pmatrix}$ &
			$\begin{pmatrix}
                4 & 4 & 0\\
                4 & 4 & 0\\
                4 & 4 & 0
			\end{pmatrix}$ \\
            \hline
			$\begin{pmatrix}
                1 & 2 & 3\\
                1 & 2 & 3
			\end{pmatrix}$ &
			$\begin{pmatrix}
                -1 & -3\\
                -2 & -2\\
                -3 & -1
			\end{pmatrix}$ &
			$\begin{pmatrix}
                -14 & -10\\
                -14 & -10
			\end{pmatrix}$ \\
            \hline
			$\begin{pmatrix}
                3 & -5\\
                1 & -2
			\end{pmatrix}$ &
			$\begin{pmatrix}
                2 & -5\\
                1 & -3
			\end{pmatrix}$ &
			$\begin{pmatrix}
			    1 & 0\\
                0 & 1
			\end{pmatrix}$ \\
            \hline
			$\begin{pmatrix}
                1 & 2 & 3\\
                1 & 2 & 3\\
                1 & 2 & 3
			\end{pmatrix}$ &
			$\begin{pmatrix}
                0 & 0 & 0\\
                0 & 0 & 0\\
                0 & 0 & 0
			\end{pmatrix}$ &
			$\begin{pmatrix}
                0 & 0 & 0\\
                0 & 0 & 0\\
                0 & 0 & 0
			\end{pmatrix}$ \\
            \hline
			$\begin{pmatrix}
                1 & 2 & 3\\
                1 & 2 & 3\\
                1 & 2 & 3
			\end{pmatrix}$ &
			$\begin{pmatrix}
                1 & 0 & 0\\
                0 & 1 & 0\\
                0 & 0 & 1
			\end{pmatrix}$ &
			$\begin{pmatrix}
                1 & 2 & 3\\
                1 & 2 & 3\\
                1 & 2 & 3
            \end{pmatrix}$ \\
            \hline
		\end{tabular}
	\end{center}
\end{table}

Алгоритмы, реализованные в данной лабораторной работе, функциональные тесты прошли успешно.


\section{Вывод}

В данном разделе были реализованы алгоритмы умножения матриц: стандартный, 
Винограда и оптимизированный Винограда. Также для проверки правильности работы программы были выделены классы эквивалентности тестов.
