\chapter{Технологическая часть}

В данном разделе описаны требования к программному обеспечению, средства
реализации, приведены листинги кода и данные, на которых будет проводиться
тестирование.

Программа должна предоставлять следующие возможности:
\begin{itemize}[left=\parindent]
    \item выбор режима работы: для единичного эксперимента и замер времени работы алгоритмов умножения матриц;
    \item в режиме единичного эксперимента ввод размеров и содержимого каждой
          матрицы и вывод полученного разными алгоритмами произведений;
    \item в режиме замера времени происходит замер времени при различных
          размерах матриц и построение графиков по полученным данным.
\end{itemize}

\section{Средства реализации}

Для реализации данной лабораторной работы Go, так как имеет простой синтаксис, 
большую библиотеку.

Замеры времени проводились при помощи функции getThreadCpuTimeNs написанной на C, 
подключенной с помощбю CGO.

\section{Реализация алгоритмов}

В данном подразделе представлены листинги кода ранее описанных алгоритмов:
\begin{itemize}[left=\parindent]
    \item стандартный алгоритм умножения матриц (листинг \ref{lst:std});
    \item алгоритм Винограда (листинги \ref{lst:Winograd});
    \item оптимизированный алгоритм Винограда (листинги
          \ref{lst:optWinMulH}-\ref{lst:WinogradOpt}).
\end{itemize}

\noindent
\begin{minipage}{\linewidth}
\begin{lstinputlisting}[
	caption={Реализация стандартного алгоритма умножения матриц},
	label={lst:std},
	linerange={94-106}
]{../algorithms/algorithms.go}
\end{lstinputlisting}
\end{minipage}

\begin{lstlisting}[label=lst:Winograd,caption=Функция нахождения расстояния Левенштейна-Дамерау с помощью рекурсии]
func MulVinogradovA(matrixA, matrixB [][]int, a, b, c int) [][]int {
    row := make([]int, a)
    column := make([]int, c)
    matrixC := make([][]int, a)

    for i := 0; i < a; i++ {
        matrixC[i] = make([]int, c)
        for j := 0; j < b / 2 ; j++ {
            row[i] = row[i] + matrixA[i][2 * j] * matrixA[i][2 * j + 1]
        }
    }

    for i := 0; i < c; i++ {
        for j := 0; j < b / 2; j++ {
            column[i] = column[i] + matrixB[2 * j][i]*matrixB[2 * j + 1][i] 
        }
    }

    for i := 0; i < a; i++ {
        for j := 0; j < c; j++ {
            matrixC[i][j] = -row[i] - column[j]
            for k := 0; k < b / 2; k++ {
                matrixC[i][j] = matrixC[i][j] + (matrixA[i][2 * k + 1] + matrixB[2 * k][j]) *
                    (matrixA[i][2 * k] + matrixB[2 * k + 1][j])
            }
        }
    }

    if (b % 2 == 1) {
        for i := 0; i < a; i++ {
            for j := 0; j < c; j++ {
                matrixC[i][j] = matrixC[i][j] + matrixA[i][b - 1] * matrixB[b - 1][j]
            }
        }
    }   

    return matrixC
}
\end{lstlisting}

\begin{lstlisting}[label=lst:WinogradOpt,caption=Функция нахождения расстояния Левенштейна-Дамерау с помощью рекурсии]
func MulVinogradovB(matrixA, matrixB [][]int, a, b, c int) [][]int {
    row := make([]int, a)
    column := make([]int, c)
    matrixC := make([][]int, a)

    num := b / 2

    for i := 0; i < a; i++ {
        matrixC[i] = make([]int, c)
        buf := 0
        for j := 0; j < num ; j++ {
            buf += matrixA[i][j << 1] * matrixA[i][j << 1 + 1]
        }
        row[i] = buf
    }

    for i := 0; i < c; i++ {
        buf := 0
        for j := 0; j < num; j++ {
            buf += matrixB[j << 1][i] * matrixB[j << 1 + 1][i] 
        }
        column[i] = buf
    }

    for i := 0; i < a; i++ {
        for j := 0; j < c; j++ {
            buf := -row[i] - column[j]
            for k := 0; k < num; k++ {
                buf += (matrixA[i][k << 1 + 1] + matrixB[k << 1][j]) *
                    (matrixA[i][k << 1] + matrixB[k << 1 + 1][j])
            }
            matrixC[i][j] = buf
        }
    }

    if (b % 2 == 1) {
        l := b - 1
        for i := 0; i < a; i++ {
            for j := 0; j < c; j++ {
                matrixC[i][j] += matrixA[i][l] * matrixB[l][j]
            }
        }
    }   

    return matrixC
}
\end{lstlisting}

\section{Вывод}

В данном разделе были реализованы алгоритмы умножения матриц: стандартный, 
Винограда и оптимизированный Винограда. Также были написаны тесты для каждого класса 
эквивалентности, описанного в конструкторском разделе.